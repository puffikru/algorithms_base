\documentclass[a4paper,12pt]{article} % добавить leqno в [] для нумерации слева

%%% Работа с русским языком
\usepackage{cmap}					% поиск в PDF
\usepackage{mathtext} 				% русские буквы в фомулах
\usepackage[T2A]{fontenc}			% кодировка
\usepackage[utf8]{inputenc}			% кодировка исходного текста
\usepackage[english,russian]{babel}	% локализация и переносы

%%% Дополнительная работа с математикой
\usepackage{amsmath,amsfonts,amssymb,amsthm,mathtools} % AMS
\usepackage{icomma} % "Умная" запятая: $0,2$ --- число, $0, 2$ --- перечисление

%% Номера формул
\mathtoolsset{showonlyrefs=true} % Показывать номера только у тех формул, на которые есть \eqref{} в тексте.

%% Шрифты
\usepackage{euscript}	 % Шрифт Евклид
\usepackage{mathrsfs} % Красивый матшрифт
\usepackage[utf8]{inputenc}

\usepackage{listings}
%% Свои команды
\DeclareMathOperator{\sgn}{\mathop{sgn}}

%% Перенос знаков в формулах (по Львовскому)
\newcommand*{\hm}[1]{#1\nobreak\discretionary{}
{\hbox{$\mathsurround=0pt #1$}}{}}

%%% Заголовок
\title{4.7.7}
\author{Igor Bulakh}
\date{\today}

\begin{document}

\maketitle

\section{Introduction}

$$\frac{\sum v_i}{\sum w_i} - \frac{\sum v_j}{\sum w_j} > 0$$

$$
V_{big} := \sum v_i := V + v_a\quad
W_{big} := \sum w_i := W + w_a\
$$
$$
V_{small} := \sum v_j := V + v_b\quad
W_{small} := \sum w_j := W + w_b\
$$

$$\frac{\sum v_i}{\sum w_i} - \frac{\sum v_j}{\sum w_j} > 0$$  multiply by $${\sum w_i} \cdot {\sum w_j}$$

$$\sum v_i \cdot \sum w_j - \sum v_j \cdot \sum w_i > 0$$

$$(V+v_i)(W+w_j) - (V + v_j)(W + w_i) > 0$$

$$V W + V w_j + v_i W + v_i w_j - V W - V w_i - v_j W - v_j w_i > 0$$

$$V(w_j - w_i) + W(v_i - v_j) + v_i w_j - v_j w_i > 0$$

$$v_i w_j = v_i (w_j - w_i + w_i) = v_i (w_j - w_i) + v_i w_i$$

$$V(w_j - w_i) + W(v_i - v_j) + v_i (w_j - w_i) + v_i w_i - v_j w_i > 0$$

$$(w_j - w_i)(V + v_i) + W(v_i - v_j) + w_i(v_i - v_j) > 0$$

$$(w_j - w_i)(V + v_i) + (v_i - v_j)(W + w_i) > 0$$

$$(w_j - w_i)V_{big} + (v_i - v_j)W_{big} > 0$$

\section{Code}

\subsection{Input/Output}

\textbf{Input:}
\begin{lstlisting}
6 2
6792 19949
22986 3872
28903 6506
2514 1990
23158 5029
2101 23700
\end{lstlisting}
\textbf{Output:}
\begin{lstlisting}
5.18414
\end{lstlisting}

\subsection{Algorithm}

$\displaystyle \frac{\displaystyle \sum v_i}{\displaystyle \sum w_i} - \frac{\displaystyle \sum v_j}{\displaystyle \sum w_j} > 0$ \\
\begin{flushleft}
$V_{big} = 86455$ \\
$W_{big} = 61046$ \\
$V = 29778$ \\
$W = 23821$ \\
\vspace{5mm}
\textbf{v and w devision results:}\newline \smallskip
$v_1 / w_1 = 0.34046819$ \\
$v_2 / w_2 = 5.93646694$ \\
$v_3 / w_3 = 4.4425146$ \\
$v_4 / w_4 = 1.26331658$ \\
$v_5 / w_5 = 4.60489163$ \\
$v_6 / w_6 = 0.08864979$ \\

$\displaystyle \frac{\sum v_i}{\sum w_i} = \frac{29778}{23821} = 1.25007346$ \\
\vspace{5mm}
Теперь вместо $v_2$ и $w_2$ выбираем следующую по порядку пару. \smallskip
$\sum v_j = 6792 + 28903 = 35695$ \\
\smallskip
$\sum w_j = 19949 + 6506 = 26455$ \\ \bigskip
$\displaystyle \frac{\sum v_j}{\sum w_j} = \frac{35695}{26455} = 1.34927235$ \\
\smallskip
Проверяем: \\
\smallskip
$1.25007346 - 1.34927235 = -0.09919889$ \quad \\
Следовательно этот элемент нам не подходит \\
\bigskip
Переходим к следующей итерации: \\
$\sum v_j = 6792 + 2514 = 9306$ \\
\smallskip
$\sum w_j = 19949 + 1990 = 21939$ \\ \bigskip
$\displaystyle \frac{\sum v_j}{\sum w_j} = \frac{9306}{21939} = 0.42417612$ \\
\smallskip
Проверяем: \\
\smallskip
$1.25007346 - 0.42417612 = 0.82589734 > 0$ \\
\bigskip
Считаем новое значение отношения $V$ к $W$ \\
\smallskip
$9306 / 21939$
\end{flushleft}
\vspace{1cm}




% \begin{lstlisting}[language=C++]
% #include <iostream>
%
% using namespace std;
%
% int C(int n, int k)
% {
%     if (k == 0 || k == n)
%         return 1;
%     return C(n - 1, k - 1) * n / k;
% }
%
% int main()
% {
%     int n, k;
%     cout << "Input n: ";
%     cin >> n;
%     cout << "Input k: ";
%     cin >> k;
%     cout << "test" << endl;
%     return 0;
% }
% \end{lstlisting}
%
\end{document}
